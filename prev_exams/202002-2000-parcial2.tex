%%% Template originaly created by Karol Kozioł (mail@karol-koziol.net) and modified for ShareLaTeX use

\documentclass[addpoints]{exam}

\usepackage[T1]{fontenc}
\usepackage[utf8]{inputenc}
\usepackage{graphicx}
\usepackage{xcolor}

\renewcommand\familydefault{\sfdefault}
\usepackage{tgheros}

\usepackage{amsmath,amssymb,amsthm,textcomp}
\usepackage{enumerate}
\usepackage{multicol}
\usepackage{tikz}
\usepackage[spanish]{babel}
\usepackage{enumitem}

\usepackage{geometry}
\geometry{left=25mm,right=25mm,%
bindingoffset=0mm, top=20mm,bottom=20mm}


\linespread{1.3}

\newcommand{\linia}{\rule{\linewidth}{0.5pt}}

% custom theorems if needed
\newtheoremstyle{mytheor}
{1ex}{1ex}{\normalfont}{0pt}{\scshape}{.}{1ex}
{{\thmname{#1 }}{\thmnumber{#2}}{\thmnote{ (#3)}}}
  
  \theoremstyle{mytheor}
  \newtheorem{defi}{Definition}
  
  % my own titles
  \makeatletter
  \renewcommand{\maketitle}{
    \begin{center}
    \vspace{2ex}
    {\huge \textsc{\@title}}
    \vspace{1ex}
    \\
    \linia\\
    \@author \hfill \@date
    \vspace{4ex}
    \end{center}
  }
  \makeatother
  %%%
  
  % custom footers and headers
  \lfoot{Assignment \textnumero{} 5}
  \cfoot{}
  \rfoot{Page \thepage}
  \renewcommand{\theenumi}{\Alph{enumi}}
  %
  
  % code listing settings
  \usepackage{listings}
  \lstset{
    language=Python,
    basicstyle=\ttfamily\small,
    aboveskip={1.0\baselineskip},
    belowskip={1.0\baselineskip},
    columns=fixed,
    extendedchars=true,
    breaklines=true,
    tabsize=4,
    prebreak=\raisebox{0ex}[0ex][0ex]{\ensuremath{\hookleftarrow}},
    frame=lines,
    showtabs=false,
    showspaces=false,
    showstringspaces=false,
    keywordstyle=\color[rgb]{0.627,0.126,0.941},
    commentstyle=\color[rgb]{0.133,0.545,0.133},
    stringstyle=\color[rgb]{01,0,0},
    numbers=left,
    numberstyle=\small,
    stepnumber=1,
    numbersep=10pt,
    captionpos=t,
    escapeinside={\%*}{*)}
  }
  
  %%%----------%%%----------%%%----------%%%----------%%%
  
  \begin{document}
  
  \title{Parcial 2 - Estadística II}
  
  \author{ITAM, Primavera 2020}
  
  \date{13/10/2020}
  
  \maketitle
  
  \section*{Instrucciones}
  
 El examen consta de una sección. Se debe desarrollar el problema planteado y cuidar la formalidad al escribir los resultados, ya que es parte de la calificación del problema. En caso de no tener el desarrollo de la pregunta, o bien se llegue a la respuesta sin una justificación se anulará la respuesta. 
  
  \vspace{10pt}
  
El examen tiene una duración de 1:45 horas. Cualquier práctica fraudulenta será sancionada de acuerdo al reglamento del departamento. La hora de entrega es 21:50 al correo: salvador.garcia.gonzalez@itam.mx

  \section*{Seccion A: Preguntas a desarrollar (100 pts)}
  
  
  \begin{questions} 



  \question N (15 pts) Sea Y una variable aleatoria Exponencial con media $\theta$ y varianza $\theta^2$. Determine el sesgo del estimador:
  
  $$\hat{\theta} = \sum_{i}^{n} Y_i (Y_i-1)$$

  \question (15 pts) Los cuidadores del una reserva ecológica de jaguares observaron las zonas de distribución estacional de los felinos. 5 Jaguares observados en primavera mostraron zonas de distribución de 8.0, 12.1, 8.1, 18.2 y 31.7 hectáreas cuadrádas. Por otra parte, 4 Jaguares diferentes se observaron en verano con zonas de 102.0, 81.7, 54.7 y 50.7 hectareas cuadradas. 
  
  \begin{enumerate}
  \item Calcule la diferencia entre zonas de distribución medias en primavera y verano con un intervalo de 95\%. 
  \item ¿Que supuestos realizó debe realizar?
  \end{enumerate}
  
\question (15 pts) Para probar un medicamento es deseable realizar verificaciones de la variabilidad de lectura producidas en las muestras. Se desea medir la variabilidad de la impureza de un medicamento que mostró 5 lecturas de 9.32, 9.48, 9.48, 9.70 y 9.26
  
  \begin{enumerate}
  \item Estime la varianza poblacional $\sigma^2$ para las lecturas usando un intervalo de confianza del 90\%
  \end{enumerate}
  
    \question (15 pts) Determine el sesgo de la pooled variance $S_p^2 = \frac{(n_1-1) S_1^2 + (n_2-1) S_2^2}{n_1 + n_2 -2}$. Suponga que $S_1^2$ y $S_2^2$ son los estimadores sesgados de la varianza, cuya esperanza es $\frac{n-1}{n} \sigma^2$ (Suponer que ambas poblaciones tienen la misma varianza)

    \question (10 pts) Un intervalo de confianza es insesgado si el valor esperado del punto medio del intervalo es igual al parámetro estimado. Determine si el intervalo de confianza para $\mu$ con $\sigma^2$ conocida es un intervalo de confianza insesgado. 
  
  \question (15 pts) Se desea estimar la proporción de radioescuchas que escucharon el anuncio de un producto. Se entrevistó a 2300 radioescuchas y resulto que 1974 de ellos lo habían escuchado.
  
  
  \begin{enumerate}
  \item Encuentre el IC de 95\% para la proporción de todos los radioescuchas que han escuchado la publicidad.
  
  \item Obtenga el tamaño de muestra requerido para que el intervalo del inciso a) tenga una longitud máxima de 10\% con la misma confianza.
  \end{enumerate}
  
  
  \question (15 pts) Los costos de manufactura de un automovil difieren de uno de otro dependiendo el número de obreros contratados. Se necesita tener una ganancia promedio por arriba de \$8,500 pesos por automovil para alganzar el plan anual. Las ganancias por automovil para los últimos 5 automóviles son \$8,760, \$6,370, \$9,620, \$8,200, \$10,350 respectivamente (Consideralas como muestra aleatoria)
 \begin{enumerate}
 \item Encuentra un intervalo de confianza del 95\% para el promedio de la ganancia por automovil.
 
\item Con la información del intervalo, ¿Es razonable pensar que se está teniendo el nivel de ganancia deseado?

\end{enumerate}
  
  
  
 
 
  \end{questions}
  \end{document}
  