%%% Template originaly created by Karol Kozioł (mail@karol-koziol.net) and modified for ShareLaTeX use

\documentclass[addpoints]{exam}

\usepackage[T1]{fontenc}
\usepackage[utf8]{inputenc}
\usepackage{graphicx}
\usepackage{xcolor}

\renewcommand\familydefault{\sfdefault}
\usepackage{tgheros}

\usepackage{amsmath,amssymb,amsthm,textcomp}
\usepackage{enumerate}
\usepackage{multicol}
\usepackage{tikz}
\usepackage[spanish]{babel}
\usepackage{enumitem}

\decimalpoint

\usepackage{geometry}
\geometry{left=25mm,right=25mm,%
bindingoffset=0mm, top=20mm,bottom=20mm}


\linespread{1.3}

\newcommand{\linia}{\rule{\linewidth}{0.5pt}}

% custom theorems if needed
\newtheoremstyle{mytheor}
{1ex}{1ex}{\normalfont}{0pt}{\scshape}{.}{1ex}
{{\thmname{#1 }}{\thmnumber{#2}}{\thmnote{ (#3)}}}
  
  \theoremstyle{mytheor}
  \newtheorem{defi}{Definition}
  
  % my own titles
  \makeatletter
  \renewcommand{\maketitle}{
    \begin{center}
    \vspace{2ex}
    {\huge \textsc{\@title}}
    \vspace{1ex}
    \\
    \linia\\
    \@author \hfill \@date
    \vspace{4ex}
    \end{center}
  }
  \makeatother
  %%%
  
  % custom footers and headers
  \lfoot{Assignment \textnumero{} 5}
  \cfoot{}
  \rfoot{Page \thepage}
  \renewcommand{\theenumi}{\Alph{enumi}}
  %
  
  % code listing settings
  \usepackage{listings}
  \lstset{
    language=Python,
    basicstyle=\ttfamily\small,
    aboveskip={1.0\baselineskip},
    belowskip={1.0\baselineskip},
    columns=fixed,
    extendedchars=true,
    breaklines=true,
    tabsize=4,
    prebreak=\raisebox{0ex}[0ex][0ex]{\ensuremath{\hookleftarrow}},
    frame=lines,
    showtabs=false,
    showspaces=false,
    showstringspaces=false,
    keywordstyle=\color[rgb]{0.627,0.126,0.941},
    commentstyle=\color[rgb]{0.133,0.545,0.133},
    stringstyle=\color[rgb]{01,0,0},
    numbers=left,
    numberstyle=\small,
    stepnumber=1,
    numbersep=10pt,
    captionpos=t,
    escapeinside={\%*}{*)}
  }
  
  %%%----------%%%----------%%%----------%%%----------%%%
  
  \begin{document}
  
  \title{Parcial 3 - Estadística II 20:00 }
  
  \author{ITAM, Otoño 2020}
  
  \date{17/11/2020}
  
  \maketitle
  
  \section*{Instrucciones}
  
El examen consta de dos secciones. La primer sección pondera 50\% de la calificación del tercer parcial y consiste en 20 preguntas de Falso/Verdadero a través de la plataforma Socrative. En pantalla se mostrará la URL. Para esta sección cada pregunta correcta cuenta 0.25 puntos, mientras que cada incorrecta resta 0.25 puntos, las preguntas en blanco no restan ningún punto. Es importante notar que una vez contestada la pregunta no será posible regresar a ella. Al final se proporcionan. Al final, se tienen dos 3 preguntas bonus que se contarán como extra. 
 
 \vspace{10pt}
 
La segunda sección consta de 3 preguntas abiertas. En estas se proporcionan la mayor parte de los datos para poder realizar una resolución rápida de la pregunta. Se debe desarrollar el problema planteado y cuidar la formalidad al escribir los resultados, ya que es parte de la calificación del problema. En caso de no tener el desarrollo de la pregunta, o bien se llegue a la respuesta sin una justificación se anulará la respuesta. 

\vspace{10pt}
  
El examen tiene una duración de 1:45 horas. \textbf{Cualquier práctica fraudulenta será sancionada de acuerdo al reglamento}. La hora de entrega es 21:50 al correo: salvador.garcia.gonzalez@itam.mx

\section*{Seccion A: Preguntas a desarrollar (100 pts)}
  
\begin{questions} 

\question 5.2.20 (15 pts) En una fábrica se cambió el sistema de compensación de los empleados para disminuir el número de fallas en la producción, por lo que se desea saber si fue exitoso. Se seleccionaron 12 empleados y se tienen los siguientes datos del número de fallas (Sea D la diferencia entre el número de fallas después del cambio menos el número de fallas antes del cambio): $\bar{D} = -0.66$, $S_D = 1.97$.

\begin{enumerate}
\item Encuentre $H_0$ y $H_1$ correspondiente
\item Resuelva la prueba de hipótesis con $\alpha=0.05$
\item Muestre en una gráfica la región de rechazo y el estadístico de prueba correspondiente
\item Inteprete el resultado
\end{enumerate}


\question 5.2.20 (15 pts)

La Secretaria de Salud decide decide comprar una cantidad importante de ventiladores. El productor de estos ventiladores afirma que tienen una vida promedio de 5 años, con una varianza menor a 2 años. La Secretaria decide adquirirlos solamente si tienen una desviación estandar menor 1.4142 años de duración. Se seleccionan al azar 51 ventiladores y se obtiene $S^2 = 1.5$. Usted está convencido que el productor está diciendo la verdad, así que desea formular una prueba de hipótesis para apoyarlo. (Se sabe que $\chi_{50,.025} = 71.42$, $\chi_{50,.975} = 32.35$, $\chi_{50,.05} = 67.50$, $\chi_{50,.95} =34.76 $)

\begin{enumerate}
\item Formule las hipótesis $H_0$ y $H_1$ apropiadas al problemas
\item Pruebe con un nivel de significancia de $\alpha = 0.025$
\item Muestre en una gráfica la región de rechazo y el estadístico de prueba correspondiente
\item Inteprete el resultado
\end{enumerate}

\question 5.2.21 (15 pts)
La dirección del ITAM desea saber si existe una diferencia estadística entre la varianza de las calificaciones de las clases en linea comparándolas con las clases presenciales. Para esto selecciona las calificaciones que tuvieron 12 alumnos previo a la pandemia (X: clases presenciales), contra 16 en la situación actual (Y: clase en linea). Los datos obtenidos son los siguientes: $\bar{X} = 2.49$, $\bar{Y} = 2.93$, $S_{X} = 1.46$, $S_{Y} = 1.11$

Suponiendo que ambas muestras provienen de poblaciones normales, probar las siguiente hipótesis con $\alpha = 0.05$:

\begin{enumerate}
\item Formule las hipótesis $H_0$ y $H_1$ apropiadas al problemas
\item Pruebe con un nivel de significancia de $\alpha = 0.05$
\item Muestre en una gráfica la región de rechazo y el estadístico de prueba correspondiente
\item Inteprete el resultado
\end{enumerate}


\end{questions}
\end{document}
  

  